%Written by: Aaron Stillmaker
%September 11, 2018
%ECE 186A - Senior Design
%
%This is a template you can use for your Project Descriptions, though you will need to change a good portion of it for your own needs.

\documentclass{IEEEtran}					%Everything is default, which is journal, 10pt font, and final draft
\usepackage{dtk-logos}						%This let me do the cool BibTeX logo, you shouldn't need this line. Add additional packages here:

\title{\vspace{2in}Smart Whiteboard Eraser}	%the \vspace is moving the title down from the top of the page
\author{ECE 186A - Senior Design I \\ 			%I am jamming all of the stuff I want on the title page into the Author spot, I know, 
												%this isn't elegant.
	Fall 2018 - Dr. Stillmaker \\ 				%Note the \\ line returns to make sure to put each one on a new line
	\vspace{12pt} 								%I put in some space to get the due date listed a little lower
	\textbf{Date:} Friday, September 28, 2018  \\ 
	\vspace{2in}								%You may need to mess with this space, this space is above the signature lines.
	\vspace{6pt}
	\textbf{Project Manager:} Heather Libecki			%Uncomment this section and fill in your team names and your technical advisor's name.
	\vspace{12pt}
	\underline{\hspace{3in}}\par					%You are welcome to have more than one technical advisor if you wish..
	\textbf{Team Member:} Chris Quesada
	\vspace{12pt}
	\underline{\hspace{3in}}\par
	\textbf{Team Member:} Juan Colin
	\vspace{12pt}
	\underline{\hspace{3in}}\par
	\textbf{Technical Advisor:} Dr. Kulhandjian
	\underline{\hspace{3in}} \\
	\vspace{12in}}								%You will need to mess with this space, this space is after the signature lines.  The idea
												%is to make the title page by itself.  I know, this isn't elegant either, but IEEE %formatting doesn't make title pages

\begin{document}
	\thispagestyle{empty}						%This removes the page number from the title page.
	
	\maketitle									%This generates the title from the information given above.
	
	
	\section{Introduction}
	Our project, the Smart Eraser, is an automatic whiteboard eraser. The main deliverable of this project will be an eraser which can move left-to-right on a track, and up-and-down on a bar attached to the track. This eraser will be able to automatically detect on a whiteboard where there are markings through the use of a camera and image-detection programming. The camera will send the coordinates of the markings to the eraser, which will then find the quickest route to erase the markings, before returning to its’ stand-by position. The eraser will also be able to detect the presence of a person, resulting in the immediate termination of whatever process it was carrying out.
	Heather Libecki is responsible for being the project manager, the wireless communication between the camera and the imbedded system, and she will also be contributing to the completion of the image-processing program which will detect where the markings on the board are. Chris Quesada is responsible for developing the microcontroller code, particularly the physical movement protocols of the system, using the coordinates specified by the camera, and he will also be contributing to the completion of the image-processing program which will detect if a person is in front of the board. Juan Colin is responsible for the creation and execution of the physical, mechanical system that the eraser will be attached to, including the wiring of the system itself, and he will also be contributing to the research and implementation of the power system. All three of the team members will assist each other with the initial research, then break off into their own specialties as the project completion progresses.
	
	\section{Background}
	The main goal of this project is to create a fully-functional automatic eraser that detects where it needs to erase as well as if there is a person in front of the board. The following objectives will need to be accomplished in order to obtain this goal: design a track system for the eraser, design a power system for the track, design a control panel/interface, create a compact unit that can easily be moved on the track, create an image-processing program in Matlab, create the path/coordinate system for the eraser to follow, communicate the created path to the Raspberry Pi, and detect if a person is standing in front of the board.
	The following list contains the strengths and weaknesses that were found for each member of the group.\\\\
	Heather Libecki\\
	$\bullet$ Strengths: programming (assembly, Matlab), mathematics, debugging, circuit implementation, problem solving, technical writing, public speaking\\
	$\bullet$ Weaknesses: programming (verilog, Python), circuitry design, coding algorithms, power systems, Github\\
	Chris Quesada\\
	$\bullet$ Strengths: programming (assembly, verilog, Matlab) embedded systems, algorithms, brainstorming\\
	$\bullet$ Weaknesses: programming (Python), circuitry design, mathematics, Github, public speaking\\
	Juan Colin\\
	$\bullet$ Strengths: electrical systems, circuitry design, problem solving, power systems, problem solving, public speaking/relations\\
	$\bullet$ Weaknesses: programming (assembly, verilog, Matlab, Python), Github, technical writing and spelling\\
	The background information that will need to be researched in order to implement this project is: image-processing programs, translating data from image-processing into coordinates, wireless communication between separate devices, Simulink in Matlab, Github repositories, Python usage, connecting physical devices for the track of the eraser, language and usage of Raspberry Pi, and other research as necessary. The specific courses that this group has taken that will be used in the implementation of this project are: Embedded Systems (ECE 178), Introduction to Electrical and Computer Engineering Tools (ECE 72), Computer Networks (ECE 146), Principles of Electrical Circuits (ECE 90), Microprocessor Architecture and Programming (ECE 118), Digital Logic Design (ECE 85), and Switching Theory and Logical Design (ECE 106).
	
	\section{Project Description}
	The Smart Eraser will use a microcontroller to process images obtained through communication with a camera in order to detect where markings are on a whiteboard. The coordinates of the marks will be detected through the aforementioned image-processing, which will determine where the eraser moves on the whiteboard. There will also be a separate protocol which will detect if there is a person standing in front of the board. The major components that may potentially be needed for this project are in the following list (prices listed are a rough estimate based on preliminary internet research).\\\\
$\bullet$ Standard Eraser (3 dollars) \\
$\bullet$ Standard Whiteboard (40 dollars)\\
$\bullet$ Stepper motors (60 dollars x2)\\
$\bullet$ H-bridge (2 dollars)\\
$\bullet$ Microcontrollers: Raspberry Pi (35 dollars)\\
$\bullet$ HD Camera (largely varying)\\
$\bullet$ Metal roller track, mounting system (largely varying)\\
$\bullet$ Pulley system, wheels/gears (largely varying)\\
$\bullet$ Various wires and connections (10 dollars)\\

	
	\section*{References}

\end{document}
